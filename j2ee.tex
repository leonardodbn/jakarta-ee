\documentclass[11pt, letterpaper, titlepage]{article}
\usepackage[utf8]{inputenc}
\usepackage{arsclassica}
\usepackage{geometry}
\usepackage{xcolor}
\usepackage{hyperref}
\usepackage{dirtree}
\usepackage{tabularx}

\geometry{a4paper,top=3cm,bottom=3cm,left=2cm,right=2cm,%
heightrounded,bindingoffset=5mm}

\newcommand{\openab}{\textcolor{red}{$<$\%} }
\newcommand{\closeab}{\textcolor{red}{\%$>$} }

\newcommand{\opencb}[1]{\textcolor{red}{$<$c:}\textcolor{cyan}{#1} }
\newcommand{\closecb}{\textcolor{red}{$>$}}

\newcommand{\closect}[1]{\textcolor{red}{$<$/c:}\textcolor{cyan}{#1}\textcolor{red}{$>$}}
\newcommand{\openct}[1]{\textcolor{red}{$<$c:}\textcolor{cyan}{#1}\textcolor{red}{$>$}}

\newcommand{\class}[1]{\textcolor{orange}{#1}}
\newcommand{\object}[1]{\textcolor{blue}{#1}}
\newcommand{\method}[1]{\textcolor{red}{#1}}
\newcommand{\propery}[1]{\textcolor{cyan}{#1}}
\newcommand{\annotation}[1]{\textcolor{cyan}{#1}}

\renewcommand{\arraystretch}{1.5} %tables extra cell space

\begin{document}
\title{Jakarta EE}
\author{Leonardo De Boni}
\maketitle
\tableofcontents
\newpage

\section{Introduzione}
Il package che ci interessa \'{e}: \textcolor{yellow}{java.net}. Le classi che ci interessano sono:
\begin{itemize}
    \item \textcolor{orange}{InetAddress}: rappresenta un indirizzo ip
    \item \textcolor{orange}{Socket}: permette di creare un client
    \item \textcolor{orange}{ServerSocket}: permette di creare un Server
\end{itemize}

\subsection{Ottenere un indirizzo ip}
Per ottenere l'indirizzo ip a partire da un nome possiamo fare cos\'{i}:\par
\textcolor{orange}{InetAddress} \textcolor{blue}{address} = \textcolor{orange}{InetAddress}.\textcolor{red}{getByName}(\textcolor{orange}{String} \textcolor{blue}{hostName});

\subsection{Connettersi al server}
Per connettersi a un server dopo aver ottenuto l'indirizzo possiamo creare un client cos\'{i}:\par
\textcolor{orange}{Socket} \textcolor{blue}{socket}  = new \textcolor{orange}{Socket}(\textcolor{orange}{InetAddress} \textcolor{blue}{adress}, \textcolor{orange}{int} \textcolor{blue}{port})\\
Dal socket possiamo ottenere un \textcolor{orange}{InputStream}:\par
\textcolor{orange}{InputStream} \textcolor{blue}{in} = \textcolor{blue}{socket}.\textcolor{red}{getInputStream}();

\section{Servlet}
Classe che mi permette di elaborare una richiesta. Per crearne uno devo estendere la classe \textcolor{orange}{HttpServlet}.
I metodi che posso sovrascrivere che corrispondono alle varie richieste sono i seguenti:
\begin{itemize}
    \item \textcolor{red}{doGet} (\textcolor{orange}{HttpServletRequest} \textcolor{blue}{req}, \textcolor{orange}{HttpServletResponse} \textcolor{blue}{res})
    \item \textcolor{red}{doPost} (\textcolor{orange}{HttpServletRequest} \textcolor{blue}{req}, \textcolor{orange}{HttpServletResponse} \textcolor{blue}{res})
    \item \textcolor{red}{doHead} (\textcolor{orange}{HttpServletRequest} \textcolor{blue}{req}, \textcolor{orange}{HttpServletResponse} \textcolor{blue}{res})
    \item \textcolor{red}{doOptions} (\textcolor{orange}{HttpServletRequest} \textcolor{blue}{req}, \textcolor{orange}{HttpServletResponse} \textcolor{blue}{res})
    \item \textcolor{red}{doTrace} (\textcolor{orange}{HttpServletRequest} \textcolor{blue}{req}, \textcolor{orange}{HttpServletResponse} \textcolor{blue}{res})
    \item \textcolor{red}{doPut} (\textcolor{orange}{HttpServletRequest} \textcolor{blue}{req}, \textcolor{orange}{HttpServletResponse} \textcolor{blue}{res})
    \item \textcolor{red}{doDelete} (\textcolor{orange}{HttpServletRequest} \textcolor{blue}{req}, \textcolor{orange}{HttpServletResponse} \textcolor{blue}{res})
\end{itemize}

\subsection{Servlet annotations}

\newpage
\section{Struttura di un progetto web}
Questa pu\'o essere la struttura semplificata di un progetto JavaEE:
\dirtree{%
.1 /src/main. 
.2 java/com/example. 
.3 ExampleServlet.java. 
.3 ExampleFilter.java. 
.3 \dots. 
.2 resources. 
.3 example.pdf. 
.3 example.png. 
.3 \dots. 
.2 webapp. 
.3 WEB-INF. 
.4 web.xml. 
.4 \dots. 
.3 index.jsp o index.html. 
.3 example.jsp o example.html. 
.3 \dots. 
}

\section{Resources}

\section{Filters}

\subsection{Filter annotations}

\section{JSP (Jakarta Server Pages)}
JSP mi permette di scrivere codice java all'interno di HTML. Per fare ci\`o posso
usare delimitatori dedicati:
\begin{itemize}
    \item \textbf{Scriptlets}: \openab \dots istruzioni java \dots \closeab
    \item \textbf{Expressions}: \openab= \dots espressioni java \dots \closeab sar\'a mostrato al client
    \item \textbf{Declarations}: \openab! \dots espressioni java \dots \closeab posso dichiarare metodi
\end{itemize}
Nelle pagine JSP ho a disposizione vari oggetti impliciti:
\begin{itemize}
    \item \textcolor{blue}{request} \textcolor{orange}{HttpServletRequest}
    \item \textcolor{blue}{response} \textcolor{orange}{HttpServletResponse}
    \item \textcolor{blue}{out} \textcolor{orange}{OutputStream} (\textcolor{blue}{response}.\textcolor{red}{getOutputStream}())
    \item \textcolor{blue}{page} \textcolor{purple}{this}
    \item \textcolor{blue}{exception} \textcolor{orange}{Throwable}
    \item \textcolor{blue}{config} \textcolor{orange}{ServletConfig}
\end{itemize}

\subsection{Expression Language}
Permette di utilizzare \textbf{expressions} pi\'u snelle:\par
\textcolor{red}{\$\{} expression  \textcolor{red}{\}}\\
Quindi per esempio\par
\openab= request.getParameter("username") \closeab\\
diventa:\par
\textcolor{red}{\$\{} \textcolor{blue}{param}.\textcolor{green}{username} \textcolor{red}{\}}\\
Inoltre ho a disposizione:
\begin{itemize}
    \item TODO
\end{itemize}

\subsection{Direttive}
Le direttive sono indicazioni che influenzano sulla struttura dell'oggetto:\par
\openab\textcolor{cyan}{@page} specifiche \closeab\par
\openab\textcolor{cyan}{@include} risorsa da includere \closeab \par
\openab\textcolor{cyan}{@taaglib} libreria di tag speciali \closeab \\
La direttiva \textcolor{cyan}{@page} mi permette di impostare delle specifiche, estendere
servlet o importare pacchetti:\\
\openab\textcolor{cyan}{@page}\par
extends="servlet"\par
import="package"\par
isErrorPage="true o false"\par
errorPage="errorPage.jsp" \closeab\\
Con la direttiva \textcolor{cyan}{@include} posso includere una risorsa. L'utilizzo pi\'u
comune \'e per includere header e footer che rimangono uguali nella maggior parte delle
pagine di un sito. Utilizzo:\par
\openab\textcolor{cyan}{@include} file="/path/to/file" \closeab \\
La direttiva \textcolor{cyan}{@taaglib} mi permette di includere librerie di tag come per
esempio jstl. Utilizzo di esempio:\par
\openab\textcolor{cyan}{@taaglib} prefix="c" uri="http://java.sun.com/jsp/jstl/core"\closeab

\subsection{Tag Library}
L'equivalente dell'if \'e:\\
\opencb{if}test="\$\{expression\}" var="nome"\closecb\par
\$\{nome\}\\
\closect{if}\\
Se volessi creare una catena if/else if/else posso usare:\\
TODO\\
Per creare un ciclo for posso fare:\\
\opencb{forEach}var="i" begin="1" end="5\closecb\par
$<$li$>$ Punto numero \$\{i\}$<$/li$>$\\
\closect{forEach}\\
Oppure posso iterare su una collection:\\
\opencb{forEach}items="\$\{collection\}" var="item"\closecb\par
$<$li$>$ \$\{item.property\} $<$/li$>$\\
\closect{forEach}

\section{Sessione}

\subsection{Session e Filter}

\section{Pattern MVC (Model View Controller)}

\newpage
\subsection{Struttura MVC}
\dirtree{%
.1 /src/main. 
.2 java/com/example/appname. 
.3 beans. 
.4 ExampleModel.java. 
.4 \dots. 
.3 controller. 
.4 filters. 
.5 ExampleFilter.java. 
.5 \dots. 
.4 servlets. 
.5 ExampleServlet.java. 
.5 \dots. 
.4 ExampleListener.java. 
.4 \dots. 
.3 exceptions. 
.4 ExampleException.java. 
.4 \dots. 
.3 persistence. 
.4 dao. 
.5 DaoFactory.java. 
.5 ExampleDao.java. 
.5 \dots. 
.4 impl. 
.5 DaoFactoryImpl.java. 
.5 ExampleDaoImpl.java. 
.5 \dots. 
.4 DbConfig.java. 
.4 DbTools.java. 
.2 resources. 
.3 example.pdf. 
.3 example.png. 
.3 \dots. 
.2 webapp. 
.3 img. 
.3 WEB-INF \DTcomment{i contenuti non sono visibili dall'esterno}. 
.4 jsp. 
.4 lib. 
.4 web.xml. 
.3 index.jsp o index.html. 
.3 example.jsp o example.html. 
.3 \dots. 
}
\newpage

\section{REST API}

\section{JSON-B}
Questa libreria mi permette di effettuare conversioni immediate da JSON a un Oggetto con
propriet\`a. Per prima cosa devo ottenere il builder:\par
\class{Jsonb} \object{builder} = \class{JsonbBuilder}.\method{create}();\\
Successivamente se devo fare una serializzazione (Object$\rightarrow$JSON):\par
\class{String} \object{json} = \object{builder}.\method{toJson}(\class{Object} \object{object});\\
Invece se devo fare una deserializzazione (JSON$\rightarrow$Object):\par
\class{Object} \object{object} = \object{builder}.\method{fromJson}(\class{String} \object{json}, \class{Object}.\propery{class});

\subsection{Annotations}
All'interno dei POJO posso usare annotations per cambiare il comportamento del builder durante
le conversioni. A disposizione ho:
\begin{itemize}
    \item \annotation{@JsonbProperty}("property-name") Applicato a un getter posso cambiare il nome di una propiet\`a nel JSON
    \item \annotation{@JsonbTransient} Ignora la propriet\`a
\end{itemize}

\section{JAX-RS}
Per impostare il path della mia Application devo prima creare una classe che estende\par
\class{jakarta.ws.rs.core.Application}\\
e poi utilizzare l'annotation:\par
\annotation{@ApplicationPath}("path")\\
Quindi tutti gli endpoint della mia API saranno mappati sotto il path che ho appena
impostato. Per impostare il path di un endpoint posso usare l'annotation:\par
\annotation{@Path}("path")

\subsection{Mappatura dei metodi}
Se volessi che un metodo di un Servlet risponda a un certo metodo HTTP posso anteporre
l'annotation corrispondente:
\begin{table}[h!]
    \begin{tabularx}{\textwidth}{| >{\centering\arraybackslash}X | >{\centering\arraybackslash}X | >{\centering\arraybackslash}X |}
        \hline
        endpoint             & metodi & altro                 \\
        \hline\hline
        \annotation{@POST}   & CREATE & \annotation{@PATCH}   \\
        \hline
        \annotation{@GET}    & READ   & \annotation{@HEAD}    \\
        \hline
        \annotation{@PUT}    & UPDATE & \annotation{@OPTIONS} \\
        \hline
        \annotation{@DELETE} & DELETE &                       \\
        \hline
    \end{tabularx}
\end{table}

\subsection{Mappatura dei parametri}
Inoltre posso mappare grazie alle annotation i parametri passati tramite url al parametro di un metodo Java.
Esempio:\par
url: http://example.com/appname/api/endpoint?parameter=example\\
Posso mappare il parametro a un attributo del mio metodo che risponde a GET:\par
public \class{Object} \method{method}(\annotation{@QueryParam}("parameter") \class{String} \object{string})\\
Se ripondo a un POST (Form) l'utilizzo \'e uguale ma il nome dell'annotation \'e:\par
\annotation{@FormParam}

\subsection{Path come parametro}
Per utilizzare il path come parametro si possono usare le seguenti annotation:\par
\annotation{@Path}("/\{parameter\}")\par
\annotation{@Path}("/\{parameter\}/endpoint")\par
public \class{Object} \method{method}(\annotation{@PathParam}("parameter") \class{String} \object{string})\\
Possiamo inoltre assegnare a un parametro un valore di default usando l'annotation:\par
\annotation{@DefaultValue}

\end{document}