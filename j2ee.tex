\documentclass[11pt, letterpaper, titlepage]{article}
\usepackage[utf8]{inputenc}
\usepackage{arsclassica}
\usepackage{geometry}
\usepackage{xcolor}
\usepackage{hyperref}
\usepackage{dirtree}

\geometry{a4paper,top=3cm,bottom=3cm,left=2cm,right=2cm,%
heightrounded,bindingoffset=5mm}


\begin{document}
\title{Jakarta EE}
\author{Leonardo De Boni}
\maketitle
\tableofcontents
\newpage

\section{Introduzione}
Il package che ci interessa \'{e}: \textcolor{yellow}{java.net}. Le classi che ci interessano sono:
\begin{itemize}
    \item \textcolor{orange}{InetAddress}: rappresenta un indirizzo ip
    \item \textcolor{orange}{Socket}: permette di creare un client
    \item \textcolor{orange}{ServerSocket}: permette di creare un Server
\end{itemize}

\subsection{Ottenere un indirizzo ip}
Per ottenere l'indirizzo ip a partire da un nome possiamo fare cos\'{i}:\par
\textcolor{orange}{InetAddress} \textcolor{blue}{address} = \textcolor{orange}{InetAddress}.\textcolor{red}{getByName}(\textcolor{orange}{String} \textcolor{blue}{hostName});

\subsection{Connettersi al server}
Per connettersi a un server dopo aver ottenuto l'indirizzo possiamo creare un client cos\'{i}:\par
\textcolor{orange}{Socket} \textcolor{blue}{socket}  = new \textcolor{orange}{Socket}(\textcolor{orange}{InetAddress} \textcolor{blue}{adress}, \textcolor{orange}{int} \textcolor{blue}{port})\\
Dal socket possiamo ottenere un \textcolor{orange}{InputStream}:\par
\textcolor{orange}{InputStream} \textcolor{blue}{in} = \textcolor{blue}{socket}.\textcolor{red}{getInputStream}();

\section{Servlet}
Classe che mi permette di elaborare una richiesta. Per crearne uno devo estendere la classe \textcolor{orange}{HttpServlet}.
I metodi che posso sovrascrivere che corrispondono alle varie richieste sono i seguenti:
\begin{itemize}
    \item \textcolor{red}{doGet} (\textcolor{orange}{HttpServletRequest} \textcolor{blue}{req}, \textcolor{orange}{HttpServletResponse} \textcolor{blue}{res})
    \item \textcolor{red}{doPost} (\textcolor{orange}{HttpServletRequest} \textcolor{blue}{req}, \textcolor{orange}{HttpServletResponse} \textcolor{blue}{res})
    \item \textcolor{red}{doHead} (\textcolor{orange}{HttpServletRequest} \textcolor{blue}{req}, \textcolor{orange}{HttpServletResponse} \textcolor{blue}{res})
    \item \textcolor{red}{doOptions} (\textcolor{orange}{HttpServletRequest} \textcolor{blue}{req}, \textcolor{orange}{HttpServletResponse} \textcolor{blue}{res})
    \item \textcolor{red}{doTrace} (\textcolor{orange}{HttpServletRequest} \textcolor{blue}{req}, \textcolor{orange}{HttpServletResponse} \textcolor{blue}{res})
    \item \textcolor{red}{doPut} (\textcolor{orange}{HttpServletRequest} \textcolor{blue}{req}, \textcolor{orange}{HttpServletResponse} \textcolor{blue}{res})
    \item \textcolor{red}{doDelete} (\textcolor{orange}{HttpServletRequest} \textcolor{blue}{req}, \textcolor{orange}{HttpServletResponse} \textcolor{blue}{res})
\end{itemize}

\subsection{Servlet annotations}

\newpage
\section{Struttura di un progetto web}
\dirtree{%
.1 /src/main. 
.2 java/com/example. 
.3 ExampleServlet.java. 
.3 ExampleFilter.java. 
.3 \dots. 
.2 resources. 
.3 example.pdf. 
.3 example.png. 
.3 \dots. 
.2 webapp. 
.3 WEB-INF. 
.4 web.xml. 
.4 \dots. 
.3 index.jsp o index.html. 
.3 example.jsp o example.html. 
.3 \dots. 
}

\section{Resources}

\section{Filters}

\subsection{Filter annotations}

\section{JSP (Jakarta Server Pages)}
JSP mi permette di scrivere codice java all'interno di HTML. Per fare ci\`o posso
usare delimitatori dedicati:
\begin{itemize}
    \item \textbf{Scriptlets}: $<$\% \dots istruzioni java \dots \%$>$
    \item \textbf{Expressions}: $<$\%= \dots espressioni java \dots \%$>$ sar\'a mostrato al client
    \item \textbf{Declarations}: $<$\%! \dots espressioni java \dots \%$>$ posso dichiarare metodi
\end{itemize}
Nelle pagine JSP ho a disposizione vari oggetti impliciti:
\begin{itemize}
    \item \textcolor{blue}{request} \textcolor{orange}{HttpServletRequest}
    \item \textcolor{blue}{response} \textcolor{orange}{HttpServletResponse}
    \item \textcolor{blue}{out} \textcolor{orange}{OutputStream} (\textcolor{blue}{response}.\textcolor{red}{getOutputStream}())
    \item \textcolor{blue}{page} \textcolor{purple}{this}
    \item \textcolor{blue}{exception} \textcolor{orange}{Throwable}
    \item \textcolor{blue}{config} \textcolor{orange}{ServletConfig}
\end{itemize}

\subsection{Expression Language}

\subsection{Direttive}

\subsection{Tag Library}

\section{Pattern MVC (Model View Controller)}

\subsection{Struttura MVC}

\end{document}